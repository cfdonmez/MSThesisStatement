\chapter*{\vspace{2 cm}\hfill{\centering ABSTRACT}\hfill}
\singlespacing



\centerline{\textbf {COMMUNICATION OPTIMIZATION}}
\centerline{\textbf {OF}}
\centerline{\textbf {RENEWABLE ENERGY SOURCES}}

\vspace{.5 cm}
\centerline{\textbf {DÖNMEZ, Fatih}}
\centerline{\textbf {Master Thesis, Dept. of Electrical and Electronics Engineering}}
\centerline{\textbf {Supervisor: Prof. Dr. Ahmet ALTUNCU}}
\centerline{\textbf {August, 2022, \pageref{LastPage} pages}}

\vspace{.5 cm}
\onehalfspacing
%Tezin Abstract kısmı buraya yazılıyor

People need the energy to improve their quality of life. Energy generation systems started with the use of fossil fuels and are still a preferred solution today. However, when the damage caused by the fossil fuels used due to the exponential increase in the energy need in the world is examined, people have started to search for clean energy sources that do less harm to the world. Since renewable energy sources are less efficient in terms of efficiency than fossil fuels, efforts have been started to focus on the efficient use of energy. Accordingly, it is vital to monitor the performance of power generation plants for efficient energy production. For this reason, to instantly observe the activities of the power plants, communication systems are used in which the data produced in the power plant is transmitted to the observation center. In this postgraduate study, the communication standards of renewable energy generation systems were examined, and simulation studies were carried out in line with the relevant standards. In the light of the data obtained from the designed simulations, it is aimed to optimize the costs of the communication solutions of the renewable energy plants planned to be established in the future.


\textbf{Keywords: } Optical Communication, Wimax, Wifi, Communication Networks, Optimization, Renewable Energy