

\chapter*{\vspace{2 cm}\hfill{\centering ÖZET}\hfill}
\singlespacing



\centerline{\textbf {DAĞITIK YENİLENEBİLİR ENERJİ ÜRETİM SİSTEMLERİ İÇİN}}
\centerline{\textbf {HABERLEŞME AĞI TASARIMI}}

\vspace{.5 cm}
\centerline{\textbf {Fatih, DÖNMEZ}}
\centerline{\textbf {Yüksek Lisans Tezi, Elektrik Elektronik Mühendisliği Anabilim Dalı}}
\centerline{\textbf {Tez Danışmanı: Prof. Dr. Ahmet ALTUNCU}}
\centerline{\textbf {Ağustos, 2022, \pageref{LastPage} sayfa}}

\vspace{.5 cm}
\onehalfspacing

Dağıtık yenilenebilir enerji üretim sistemleri başlığında değerlendirilen rüzgar ve güneş enerji santrallerinin bir arıza durumunda oluşturdukları maliyetleri düşürmek ve üretilen enerjilerin verimlerini arttırmak için gerçek zamanlı olarak takiplerinin yapılması elzem bir durumdur. Yenilenebilir enerji santrallerinde kullanılan haberleşme sistemlerinin standartları \gls{iec} ve \gls{ieee} tarafından belirlenmiştir.
Bu yüksek lisans tezinde ilgili standartlara bağlı kalarak \gls{dpu} bünyesinde dağıtık yenilenebilir enerji sistemleri kurulması durumunda gereken haberleşme ağı tasarımları yapılıp ve maliyetleri analizlenmiştir.

\begin{comment}
Bu yüksek lisans çalışmasında, dağıtık yenilenebilir enerji üretim sistemlerindeki sensör verilerinin merkezden gerçek zamanlı olarak incelenebilinmesi ve ilgili haberleşme standartlarının sağlandığı ağ

\end{comment}

\textbf{Anahtar Kelimeler: } Optik Haberleşme, Wimax, Wifi, Haberleşme Ağları, Eniyileme, Yenilenebilir Enerji 

%Tezin özet kısmını buraya yazıyoruz