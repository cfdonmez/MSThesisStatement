

\chapter*{\vspace{2 cm}\hfill{\centering ÖZET}\hfill}
\singlespacing



\centerline{\textbf {DAĞITIK YENİLENEBİLİR ENERJİ ÜRETİM SİSTEMLERİ İÇİN}}
\centerline{\textbf {HABERLEŞME AĞI TASARIMI}}

\vspace{.5 cm}
\centerline{\textbf {Fatih, DÖNMEZ}}
\centerline{\textbf {Yüksek Lisans Tezi, Elektrik Elektronik Mühendisliği Anabilim Dalı}}
\centerline{\textbf {Tez Danışmanı: Prof. Dr. Ahmet ALTUNCU}}
\centerline{\textbf {Ağustos, 2022, \pageref{LastPage} sayfa}}

\vspace{.5 cm}
\onehalfspacing


Bilginin bir noktadan başka bir noktaya veya noktalara aktarımında haberleşme sistemi kullanılır. Böylece bilgiyi üreten yapı hakkında oluşabilecek arıza durumlarının önüne geçilebilen bir ikaz sistemi kurularak, ilgili yapının çalışmasında sürdürülebilirlik sağlanmış olur.

Yenilenebilir enerji santralleri, dağıtık enerji sistemleri olarak değerlendirilir. Enerji verimleri, konvansiyonel enerji santrallerine göre düşük, arıza müdahalesinden kaynaklı maliyetleri fazladır. Yenilenebilir enerji santrallerine yapılacak yatırımın cazip olması için, enerji üretim maliyetini minimize edecek çözümlere ağırlık verilmesi gerekmektedir. Haberleşme sistemleri sayesinde, yenilenebilir enerji sistemlerindeki muhtemel arızalar için zamanında aksiyon alınarak, enerji üretim maliyetinin düşürülmesi sağlanır. 


Bu yüksek lisans tezinde, dağıtık yenilenebilir enerji üretim sistemlerinin üretim maliyetlerini minimize edebilmek için; kolayca ölçeklenebilir, geleneksel haberleşme sistemlerine göre yatırım maliyeti düşük ve düşük gecikme değerleriyle sürekli olarak müdahale edilebilir bir yapıyı destekleyecek haberleşme ağları araştırılmıştır. \gls{iec} ve \gls{ieee} tarafından belirlenen haberleşme standartları incelenmiştir. Yenilenebilir enerji santralleri hakkında temel anlamda çalışma prensipleri incelenmiştir. Kütahya Dumlupınar Üniversitesi bünyesinde dağıtık yenilenebilir enerji sistemleri kurulması durumunda haberleşme ihtiyacının çözümü için ağ tasarımları yapılmıştır. Tasarlanan ağların haberleşme performansları, yatırım maliyetleri ve \gls{iec} ve \gls{ieee} tarafından belirlenen haberleşme standartlarına uyumluluğu değerlendirilmiştir. 


\begin{comment}


Dağıtık yenilenebilir enerji üretim sistemleri başlığında değerlendirilen rüzgar ve güneş enerji santrallerinin bir arıza durumunda oluşturdukları maliyetleri düşürmek ve üretilen enerjilerin verimlerini arttırmak için gerçek zamanlı olarak takiplerinin yapılması elzem bir durumdur. Yenilenebilir enerji santrallerinde kullanılan haberleşme sistemlerinin standartları \gls{iec} ve \gls{ieee} tarafından belirlenmiştir.
Bu yüksek lisans tezinde ilgili standartlara bağlı kalarak \gls{dpu} bünyesinde dağıtık yenilenebilir enerji sistemleri kurulması durumunda gereken haberleşme ağı tasarımları yapılıp ve maliyetleri analizlenmiştir.


Günümüzde, artan enerji ihtiyaçlarının karşılanması konusunda insanlar dünyaya daha az zarar vererek enerji üretme konusuna ağırlık verdiler. Bu sebeple ülkeler yenilenebilir enerji kaynaklarını, fosil kaynaklara tercih etmeye başlamışlardır.

Yenilenebilir enerji santrallerinin doğası gereği dağıtık yapıda olmasından kaynaklı olarak bakım, onarım ve arıza müdahalelerinden kaynaklı maliyetleri fazladır. Yenilenebilir enerji kaynaklarından elde edilen enerjinin verimi henüz fosil yakıtlardan elde edilen enerji veriminden az olduğundan, enerji üretiminde yatırımsal anlamda maliyetlerin azaltıcı aksiyonların alınması elzemdir. Santraldeki yaşanması muhtemel arızaların önüne geçebilmek için santral içerisindeki donanımların faaliyetlerini gerçek zamanlı izlemek gerekmektedir. Bu sebeple haberleşme sistemlerinin yenilenebilir enerji santrallerindeki önemli büyüktür. 

Tüm sektörlerde olduğu gibi enerji sektöründe de teknolojik gelişmeler yaşan-maktadır. Haberleşme sistemlerindeki teknolojik gelişmelerin enerji sistemlerine uyarlanması sonucunda bakım, onarım ve enerji verimi faaliyetlerinin oluşturacağı maliyetlerin düşürülmesi amaçlanarak bu tez hazırlanmıştır.









\end{comment}





\textbf{Anahtar Kelimeler: } Optik Haberleşme, Wimax, Wifi, Haberleşme Ağları, Eniyileme, Yenilenebilir Enerji 

%Tezin özet kısmını buraya yazıyoruz