\chapter*{ÖNSÖZ}


Günümüzde, artan enerji ihtiyaçlarının karşılanması konusunda insanlar dünyaya daha az zarar vererek enerji üretme konusuna ağırlık verdiler. Bu sebeple ülkeler yenilenebilir enerji kaynaklarını, fosil kaynaklara tercih etmeye başlamışlardır.

Yenilenebilir enerji santrallerinin doğası gereği dağıtık yapıda olmasından kaynaklı olarak bakım, onarım ve arıza müdahalelerinden kaynaklı maliyetleri fazladır. Yenilenebilir enerji kaynaklarından elde edilen enerjinin verimi henüz fosil yakıtlardan elde edilen enerji veriminden az olduğundan, enerji üretiminde yatırımsal anlamda maliyetlerin azaltıcı aksiyonların alınması elzemdir. Santraldeki yaşanması muhtemel arızaların önüne geçebilmek için santral içerisindeki donanımların faaliyetlerini gerçek zamanlı izlemek gerekmektedir. Bu sebeple haberleşme sistemlerinin yenilenebilir enerji santrallerindeki önemli büyüktür. 

Tüm sektörlerde olduğu gibi enerji sektöründe de teknolojik gelişmeler yaşan-maktadır. Haberleşme sistemlerindeki teknolojik gelişmelerin enerji sistemlerine uyarlanması sonucunda bakım, onarım ve enerji verimi faaliyetlerinin oluşturacağı maliyetlerin düşürülmesi amaçlanarak bu tez hazırlanmıştır.

Bana daima cesaret veren, inancını ve desteğini hissettiren, araştırmalarıma yön verip değerli fikirleriyle yanımda olan sayın danışmanım Prof. Dr. Ahmet Altuncu'ya teşekkürlerimi sunarım.
